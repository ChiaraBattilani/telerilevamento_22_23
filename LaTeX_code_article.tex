% From R to Latex
% funzione(arg1, arg2, ...)
% #comment

% \function{}
% commenti in Latex

\documentclass[a4paper, 12pt]{article} %specificare il tipo di documento
\usepackage[utf8]{inputenc}
\usepackage{graphicx} %pacchetto per parti grafiche
\usepackage{color} %per colori scritte
\usepackage{hyperref} %per fare riferimenti a titoli, numeri img ecc, + \ref che è nel testo. Es. clicco su Fig. 1 e va a quella figura
\usepackage{lineno} %per numerazione righe
\usepackage{listings}
\usepackage{natbib}
\usepackage{setspace} % per lavorare su interlinea

\linenumbers %funzione in lineno

% in R: library()

\newcommand{\tr}{\textcolor{red}}

\title{My first document in LaTex} %Latex as symbol
\author{Chiara Battilani}
% \date{}

% \begin{} -> \end{}
\begin{document}

\maketitle %titolo
% \doublespacing %interlinea doppia
\tableofcontents %indice

% abstract
\begin{abstract}
Biodiversity conservation faces a methodological conundrum: Biodiversity measurement often relies on species, most of which are rare at various scales, especially prone
to extinction under global change, but also the most challenging to sample and model.
Predicting the distribution change of rare species using conventional species distribution models is challenging because rare species are hardly captured by most survey
systems. When enough data are available, predictions are usually spatially biased towards locations where the species is most likely to occur, violating the assumptions of
many modelling frameworks. Workflows to predict and eventually map rare species
distributions imply important trade-­offs between data quantity, quality, representativeness and model complexity that need to be considered prior to survey and analysis.
\end{abstract}

% keywords wof
\noindent \textbf{Keywords}: biodiversity; geology; Iceland lands; remote sensing; spatial ecology; statistics %no indentato e in grassetto

\section{Introduction}\label{sec:intro}

Since the 1920s, \textcolor{red}{aerial photography} has represented an important data source for the detection of landscape patterns and their change over time.

% colour
%\smallskip
%\bigskip
\tr{Multitemporal analysis represents a powerful method for the study of all ecological and geological processes that change over time. The literature involves several fields of study: from soil loss to natural resources assessment to vegetation and ecological dynamics.}

% subsection

\section{Study area}
The study area is the nature reserve of Poggio all’Olmo in Tuscany (Figure  \ref{fig:islanda}), Italy (11 28 26E, 42 51 45N, WGS84 Datum). It is located on the side of Mt. Amiata,
comprising 440 ha, with elevations ranging from 650 to 1016 m above mean sea level (m.s.l.) and slopes from 0 to 55.
% figure and symbols

\begin{figure}
\centering
 \includegraphics[width=0.8\textwidth]{islanda.jpg}
 \caption{This is Iceland!}
 \label{fig:islanda}
\end{figure}

%\newpage %va a pagina nuova
\section{Methods}
In this section I am going to describe the methods of this manuscript.

They are basically divided into:
\begin{itemize} %elenco puntato
 \item The formulas
 \item The code
\end{itemize}

Using numbers, they are basically divided into:
\begin{enumerate} %elenco numerato
 \item The formulas
 \item The code
\end{enumerate}

First of all I will strat with the formulas used and then I will pass to the code!

\subsection{Formulas}
Here is the formula used in this manuscript:
\begin{equation} % scrivere formule
F = G \times \frac{m_{1} \times m_{2}}{d^{2 \times \mu}}
\label{eq:newton}
\end{equation}

Let's make a more complex equation:
\begin{equation}
F = \frac{{}\sqrt[3]{G \times \frac{m_{1} \times m_{2}}{d^{2 \times \mu }}}}{-\sum{p(x) \times \log{p(x)}}}
\end{equation} 
%frazione, rad quadr, sommatoria, log

Let's put a formula directly in the main text.
We can apply this: $F=G \times m_{1}$

\subsection{Code}
Here is the code used in this manuscript:
\lstinputlisting[language=R]{latex_r_code.r} %inserire codice r

\section{Results}
As a result of this study I found that the Cadmium is present in the analyzed soil with the total amount of 15\%. These results were achieved according to Equations \ref{eq:newton} %per scrivere percentuale

\section{Discussion}
As stetaed in Section \ref{sec:intro}

Our results are in line with previous research dealing with remote sensing for biodiversity estimate \citep{Revell_2012, Potts_2022}. %citare articoli dentro le parentesi

\citet{Revell_2012} told us that blablabla... % citare tizio

Biodiversity has been predicted by remote sensing data based on e.g. heterogeneity f the landscape \citep{Massatti_2022}

\begin{thebibliography}{999} %fare bibliografia

\bibitem[Massatti and Winkler(2022)]{Massatti_2022}
Massatti, R. and Winkler, D.E. (2022), Spatially explicit management of genetic diversity using ancestry probability surfaces. Methods Ecol Evol. Accepted Author Manuscript. \url{https://doi.org/10.1111/2041-210X.13902} % inseire url

\bibitem[Potts et al.(2022)]{Potts_2022}
Potts, J.R., Börger, L., Strickland, B.K. and Street, G.M. (2022), Assessing the predictive power of step selection functions: how social and environmental interactions affect animal space use. Methods Ecol Evol. Accepted Author Manuscript. \url{https://doi.org/10.1111/2041-210X.13904}

\bibitem[Revell(2012)]{Revell_2012}
Revell, L.J. (2012), phytools: an R package for phylogenetic comparative biology (and other things). Methods in Ecology and Evolution, 3: 217-223. \url{https://doi.org/10.1111/j.2041-210X.2011.00169.x}

\end{thebibliography}

\end{document}
