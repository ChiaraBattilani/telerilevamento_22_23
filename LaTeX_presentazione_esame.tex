\documentclass{beamer}
\usepackage{listings}
\usepackage{color}
\usepackage[T1]{fontenc}

\usetheme{Szeged}
\usecolortheme{spruce}

\title{Esame Telerilevamento Geo-ecologico}
\institute{Alma Mater Studiorum - Università di Bologna\\Telerilevamento Geo-ecologico}
\author{Studente: Chiara Battilani \\ Docente: Duccio Rocchini}
\date{A.A. 2021/2022}

\begin{document}

\maketitle

\AtBeginSection[]
{	
\begin{frame}
\frametitle{Indice}
\tableofcontents[currentsection,currentsubsection]
\end{frame}
}

\AtBeginSubsection[]
{
\begin{frame}
\frametitle{Indice} 
\tableofcontents[currentsubsection]  
\end{frame}
}

\section{Scopi del progetto}

\begin{frame}{Scopi del progetto}
\frametitle{Obiettivo principale del progetto}
\begin{itemize}
\item Analizzare le variazioni che vi sono state in Bahrein per quanto riguarda l'aumento delle aree in cui sono state fatte costruzioni tra il 1987 e il 2022.
\end{itemize}
\end{frame}

\section{Inquadramento geografico}

\begin{frame}{Inquadramento geografico}
L'area analizzata rappresenta il Bahrein (nel riquadro rosso), situato nel Golfo Persico (da Google Earth).\\
\includegraphics[width=0.9\textwidth]{1_inquadramento.jpg}
\centering
\end{frame}

\section{Metodi di analisi}

\begin{frame}{Metodi di analisi}
\begin{itemize}
\item Importazione immagini e visualizzazione.
\bigskip
\item Calcolo pca (principal component analysis) delle immagini.
\bigskip
\item Rilevazione dell'aumento delle zone costruite.
\bigskip
\item Calcolo deviazione standard.
\bigskip
\item Classificazione delle immagini.
\end{itemize}
\end{frame}

\section{Importazione immagini e visualizzazione}

\begin{frame}{Script utilizzato}
\begin{tiny}
\lstinputlisting[language=R]{1_script.r}
\end{tiny}
\end{frame}

\begin{frame}{Bande Bahrein 1987, con palette cl}
\includegraphics[width=0.6\textwidth]{bahrain1987_bandeCL.png}
\centering
\end{frame}

\begin{frame}{Bande Bahrein 2022, con palette cl}
\includegraphics[width=0.6\textwidth]{bahrain2022_bandeCL.png}
\centering
\end{frame}

\begin{frame}{Comparazione immagini RGB 1987-2022}
\includegraphics[width=0.8\textwidth]{a1_a2_.png}
\centering
\end{frame}

\section{Calcolo pca (principal component analysis) delle immagini}

\begin{frame}{Script utilizzato}
(Esempio per l'anno 1987)
\bigskip
\begin{tiny}
\lstinputlisting[language=R]{2_script.r}
\end{tiny}
\end{frame}

\begin{frame}{Componenti pca 1987}
\includegraphics[width=0.6\textwidth]{bahrain1987_pca.png}
\centering
\end{frame}

\begin{frame}{Componenti pca 2022}
\includegraphics[width=0.6\textwidth]{bahrain2022_pca.png}
\centering
\end{frame}

\begin{frame}{Componente 1 del 1987 e 2022 con "viridis"}
\centering
\includegraphics[width=0.9\textwidth]{ba_1987_2022.png}
\end{frame}

\section{Rilevazione dell'aumento delle zone costruite}

\begin{frame}{Script utilizzato}
\begin{tiny}
\lstinputlisting[language=R]{3_script.r}
\end{tiny}
\end{frame}

\begin{frame}{Differenza tra la componente 1 del 1987 e del 2022}
\includegraphics[width=0.65\textwidth]{dif.png}
\centering
\end{frame}

\section{Calcolo deviazione standard}

\begin{frame}{Script utilizzato}
\begin{tiny}
\lstinputlisting[language=R]{4_script.r}
\end{tiny}
\end{frame}

\begin{frame}{Mappa deviazione standard}
\includegraphics[width=0.65\textwidth]{ds3.png}
\centering
\end{frame}

\section{Classificazione delle immagini}

\begin{frame}{Script utilizzato}
(Esempio per l'anno 1987)
\bigskip
\begin{tiny}
\lstinputlisting[language=R]{5_script.r}
\end{tiny}
\end{frame}

\begin{frame}{Mappe classificazione 1987 e 2022}
\centering
\includegraphics[width=0.5\textwidth]{b1987c.png}\includegraphics[width=0.5\textwidth]{b2022c.png}
\end{frame}

\begin{frame}{Grafico a barre del 1987}
\centering
\includegraphics[width=0.6\textwidth]{bar1987.png}
\end{frame}
\begin{frame}{Grafico a barre del 2022}
\centering
\includegraphics[width=0.6\textwidth]{bar2022.png}
\end{frame}

\section{Conclusioni}
\begin{frame}{Conclusioni}
\begin{itemize}
\item L'analisi delle immagini riferite al 1987 e al 2022, ha permesso di osservare come lo stato del Bahrein in 35 anni abbia subito, principalmente nella zona a nord, fenomeni antropici di creazione di zone adibite alla costruzione di edifici per far fronte al grande incremento della popolazione.
\bigskip
\item Dallo studio sulle immagini è stato calcolato un aumento del 3\% di tali zone.
\end{itemize}{}
\end{frame}

\begin{frame}
\centering
\bigskip
\bigskip
\huge
Grazie per l'attenzione!    
\end{frame}{}
\end{document}
