\documentclass{beamer} % beamer è la parte di latex dedicata alle presentazioni
\usepackage{color} % per avere differenti strutture (città) e colori (colonne) delle presentazioni https://mpetroff.net/files/beamer-theme-matrix/
\usetheme{Berkeley} % Berlin, Madrid, Frankfurt
\usecolortheme{crane} %dove, spruce, beetle

\title{My first presentation under beamer!}
\author{Chiara Battilani}

\begin{document}

\maketitle

%per fare sommario
\AtBeginSection[] % Do nothing for \section*
{	
\begin{frame}
\frametitle{Outline}
\tableofcontents[currentsection,currentsubsection,currentsubsubsection]
\end{frame}
}

\section{Introduction}

\begin{frame}
\frametitle{My first slide!} 
\begin{itemize} % per elenchi puntati
\item Mars' surface is covered with \textbf{craters} that formed when comets and/or asteroids impacted the red planet.
\item \pause These craters are especially numerous in the \textbf{southern highlands} of Mars. % pausa tra apparizione di una frase e l'altra, crea 2 slide
\item \pause Third sentence.
\end{itemize}
\end{frame}

\section{Methods}

\subsection{Instruments}

\begin{frame}
\frametitle{Instruments used}
\centering %centrare immagine
\includegraphics[width=0.7\textwidth]{sonda.jpg} %inserisce immagine ad una certa dimensione
\end{frame}

\begin{frame}
\frametitle{Instruments used}
\centering %centrare immagine
\includegraphics[width=0.7\textwidth]{sonda2.jpg} %inserisce immagine ad una certa dimensione
\end{frame}

\subsection{Analysis}

\begin{frame}
\frametitle{The used formula}

\centering %metto immagine assieme ad una formula
\includegraphics[width=0.3\textwidth]{euler.jpg} \\

\begin{equation}
e^{i \pi} + 1 = 0
\end{equation}
\end{frame}

\section{Results}

%inserire due immagini vicine
\begin{frame}
\frametitle{Results}
\centering
\includegraphics[width=0.4\textwidth]{sonda.jpg} %calo le dimensioni se no non ci stanno una accanto all'altra
\includegraphics[width=0.4\textwidth]{sonda.jpg}
\end{frame}

%inserire quattro immagini due sopra due sotto
\begin{frame}
\frametitle{Results}
\centering
\includegraphics[width=0.35\textwidth]{sonda.jpg} %calo le dimensioni se no non ci stanno una accanto all'altra
\includegraphics[width=0.35\textwidth]{sonda.jpg} \\ %va a capo
\smallskip %\bigskip, spazio tra img sopra e img sotto
\includegraphics[width=0.35\textwidth]{sonda.jpg}
\includegraphics[width=0.35\textwidth]{sonda.jpg}
\end{frame}

\end{document}
